\documentclass[10pt]{article}

% Lines beginning with the percent sign are comments
% This file has been commented to help you understand more about LaTeX

% DO NOT EDIT THE LINES BETWEEN THE TWO LONG HORIZONTAL LINES

%---------------------------------------------------------------------------------------------------------

% Packages add extra functionality.
\usepackage{times,graphicx,epstopdf,fancyhdr,amsfonts,amsthm,amsmath,algorithm,algorithmic,xspace,hyperref}
\usepackage[left=1in,top=1in,right=1in,bottom=1in]{geometry}
\usepackage{sect sty}	%For centering section headings
\usepackage{enumerate}	%Allows more labeling options for enumerate environments 
\usepackage{epsfig}
\usepackage[space]{grffile}
\usepackage{booktabs}
\usepackage{forest}
\usepackage{array}

% This will set LaTeX to look for figures in the same directory as the .tex file
\graphicspath{.} % The dot means current directory.

\pagestyle{fancy}

\lhead{Final Project}
\rhead{\today}
\lfoot{CSCI 334: Principles of Programming Languages}
\cfoot{\thepage}
\rfoot{Fall 2023}

% Some commands for changing header and footer format
\renewcommand{\headrulewidth}{0.4pt}
\renewcommand{\headwidth}{\textwidth}
\renewcommand{\footrulewidth}{0.4pt}

% These let you use common environments
\newtheorem{claim}{Claim}
\newtheorem{definition}{Definition}
\newtheorem{theorem}{Theorem}
\newtheorem{lemma}{Lemma}
\newtheorem{observation}{Observation}
\newtheorem{question}{Question}

\setlength{\parindent}{0cm}


%---------------------------------------------------------------------------------------------------------

% DON'T CHANGE ANYTHING ABOVE HERE

% Edit below as instructed

\begin{document}
  
\section*{Minimally Working Version}

Danny Klein, Austin Osborn

\subsection{Minimally Working Interpreter}
    \quad Our AST represnts multiple types of data: Numbers (ints), and Variables (characters) are the primitives. Our AST also has multiple combining forms, Bound is a tuple of two Numbers represnting the lower and upper bounds of the domain. Domain is a tuple of a Variable and a Bound, which is itself a tuple.
    \\ \quad Our parser recognizes data inputed in the form "x from 0 to 10." as an example of a valid domain statement, and produces an AST with x as the variable, 0 as the lower bound, and 10 as the upper bound. The evaluator takes this Domain expression and turns it into an svg file that draws the domain on the axis of our variable.

\subsection{Minimal Formal Grammar}
    \begin{center} 
        \begin{verbatim}
        <domian>   ::= <var> from <bound>.
        <bound>    ::= <num> to <num>
        <var>      ::= a | ... | z
        <num>      ::= <digit>^+ | -<digit>^+
        <digit>    ::= 0 | ... | 9
        \end{verbatim}
    \end{center}
    \quad The domain: "x from 0 to 5." would produce the following tree:
    \begin{center}
        \begin{forest}
            [\textlangle domain \textrangle
                [\textlangle var \textrangle
                    [x]
                ]
                ["from"]
                [\textlangle bound \textrangle
                    [\textlangle num \textrangle
                        [\textlangle digit \textrangle
                            [0]
                        ]
                    ]
                    ["to"]
                    [\textlangle num \textrangle
                        [\textlangle digit \textrangle
                            [5]
                        ]
                    ]
                ]
                ["."]
            ]
        \end{forest}
    \end{center}

\subsection{Minimal Semantics}

\begin{center}
    \begin{tabular} { |m{1cm}|m{4cm}|m{1cm}|m{2cm}|m{7cm}| }
        \hline
        Sytnax & Abstract Syntax & Type & Prec./ Assoc. & Meaning \\
        \hline
        x & Var of Char & Char & N/A & x is a primitive, and will represent our independent variable \\
        \hline
        n & Num of int & int & N/A & n is a primitive. We represent integers using the 32-bit integer data type (Int32). \\
        \hline
        n to m & Bound of \{lower: Num; upper: Num\} & record of int*int & N/A & n to m is a combining form of two Nums (ints) that represent the lower and upper bound of our domain. It is saved as a record with the first int as lower and the second int as upper. \\
        \hline
        x from n to m & Domain of \{var: Var; bounds: Bound\} & record of char* Bound & N/A & x from n to m is a combining form of a Var (char) and two Nums (ints) that represent the variable, lower bound, and upper bound of our domain. It is saved as a record with the Var as var and the two ints as a Bound record. \\
        \hline
    \end{tabular}
\end{center}

% DO NOT DELETE ANYTHING BELOW THIS LINE
\end{document}
